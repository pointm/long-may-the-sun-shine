\documentclass[12pt, a4paper, oneside]{ctexart}
\usepackage{amsmath, amsthm, amssymb, graphicx}
\usepackage[bookmarks=true, colorlinks, citecolor=blue, linkcolor=black]{hyperref}
\newtheorem{theorem}{定理}[section]

% 导言区

\title{我的第一个\LaTeX 文档}
\author{Dylaaan}
\date{\today}

\begin{document}

\maketitle

\tableofcontents

\section{一级标题}

\subsection{二级标题}

这里是正文.

\subsection{二级标题}

这里是正文.

$$
    \begin{bmatrix}
        a & b \\
        c & d
    \end{bmatrix}
$$
\begin{theorem}[定理名称]
    这里是定理的内容.
\end{theorem}

\begin{equation*}
    \left(\frac{xdx}{dy} -\frac{ydy}{dx}\right)^{2} \ ,\ [\vec{F} =m\vec{a}] \ ,\ \left| \frac{a}{b}\right| \ \left\Vert \frac{a}{b}\right\Vert \ \left< \frac{a}{b}\right> \left\{\sqrt{a+\sqrt{a+\sqrt{a}}}\rightarrow \infty \right\}
\end{equation*}


\begin{gather*}
    \begin{array}{ c c c }
        ! & \int _{b}^{a} f'( x) dx=f( b) -f( a)                                         & \underbrace{\frac{1}{4} W_{\mu \nu } \cdot W^{\mu \nu } -\frac{1}{4} B_{\mu \nu } B^{\mu \nu } -\frac{1}{4} G_{\mu \nu }^{a} G_{a}^{\mu \nu }}_{\mathrm{kinetic\ energies\ and\ self-interactions\ of\ the\ gauge\ bosons}} \\
          & \Vert x+y\Vert \geq \bigl|\Vert x\Vert -\Vert y\Vert \bigr|                  & \nabla \cdot \mathbf{D} =\rho \ \mathrm{and} \ \nabla \cdot \mathbf{B} =0\                                                                                                                                                  \\
          &                                                                              & \nabla \times \mathbf{E} =-\frac{\partial \mathbf{B}}{\partial t} \ \mathrm{and} \ \nabla \times \mathbf{H} =\mathbf{J} +\frac{\partial \mathbf{D}}{\partial t}                                                             \\
          & y=\frac{\sum\limits _{i} w_{i} y_{i}}{\sum\limits _{i} w_{i}} \ \ ,i=1,2...k & e=\lim\limits _{n\rightarrow \infty }\left( 1+\frac{1}{n}\right)^{n}                                                                                                                                                        \\
          &                                                                              &
    \end{array}\\
    \dot{x}_{i} =a_{i} x_{i'} -( d+a_{i0} +a_{i1}) x_{i} +rx_{i}( f_{i} -\phi )
\end{gather*}

\begin{equation}
    \begin{cases}
        \nabla \cdot \mathbf{D} = {\rho}_{V},                                         \\
        \nabla \cdot \mathbf{B} = 0,                                                  \\
        \nabla \times \mathbf{H} = \mathbf{J} +  {\partial\mathbf{D}\over\partial t}, \\
        \nabla \times \mathbf{E} = - {\partial\mathbf{B}\over\partial t}.
    \end{cases}
\end{equation}

\begin{align}
    y^{2}         & = x^{3} + ax + b          \\
    y^{2}         & = (x - a)(x - b)(x - c)   \\
    y^{2}         & = x^{3} + ax^{2} + bx + c \\
    x^{3} + y^{3} & = a
\end{align}

\begin{equation}
    a_{11}x^2 + 2a_{12}xy + a_{22}y^2 + 2a_{13}x + 2a_{23}y + a_{33} = 0
\end{equation}

\begin{align}
    \begin{cases}
        \frac{x^2}{a^2} + \frac{y^2}{b^2}  = 1 \\
        \frac{x^2}{a^2} - \frac{y^2}{b^2}  = 1 \\
        y^2                                = 2px
    \end{cases}
\end{align}
\begin{align}
    e^x        & = 1 + \frac{1}{1!} x + \frac{1}{2!} x^2 + \frac{1}{3!} x^3 + o(x^3)                                                                                                                    \\
    \ln(1 + x) & = x - \frac{1}{2} x^2 + \frac{1}{3} x^3 + o(x^3)                                                                                                                                       \\
    \sin x     & = x - \frac{1}{3!} x^3 + \frac{1}{5!}x^5+ o(x^5)                                                                                                                                       \\
    \arcsin x  & =  x+ \frac{1}{2}\cdot\frac{x^{3}} { 6 }+\frac {  1\times 3 } { 2\times4 }\cdot\frac{x^{5}} {40 }+\frac {   1\times 3\times5 } {   2\times4\times6 }\cdot\frac{x^{7}} {1120 }+o(x^{7})
\end{align}
\begin{align}
    \sin x & = \sum (-1)^n \frac{(2n + 1)!x^{2n+1}}{(2n + 1)!},      & \forall x                       \\
    \cos x & = \sum (-1)^n B_{2n} \frac{(4^n)(1 - 4^n)x^{2n}}{(2n)!}                                   \\
    \tan x & = \sum B_{2n} \frac{(-4)^n(1 - 4^n)x^{2n-1}}{(2n)!},    & \forall x : |x| < \frac{\pi}{2}
\end{align}
\begin{equation}
    \Omega = \frac{N!}{\prod a_i! \cdot }\prod (\omega_i^{a_i}) \tag{5}
\end{equation}
\begin{equation}
    \begin{split}
        \ln\Omega & = \ln(N!) - \sum_i \ln(a_i!) + \sum_i a_i \ln\omega_i                   \\
                  & \approx N(\ln N - 1) - \sum_i a_i(\ln a_i - 1) + \sum_i a_i \ln\omega_i \\
                  & = N\ln N - \sum_i a_i \ln\left(\frac{a_i}{\omega_i}\right)
    \end{split}
\end{equation}
\begin{align}
    \delta\Omega                                              & = 0,\delta^2\Omega < 0 \tag{5}                                                                                                   \\
    \frac{\delta\Omega}{\Omega} = \delta(\ln \Omega)          & = -\sum_i \ln \left( \frac{a_i}{\omega_i} \right) \delta a_i - \sum_i \ln \left( \frac{a_i}{\omega_i} \right) \delta a_i \tag{6} \\
    \sum_i \ln \left( \frac{a_i}{\omega_i} \right) \delta a_i & = 0 \tag{7}
\end{align}

\begin{gather}
    \begin{aligned}
        \chi_{\pm} & = - N_{\mathrm{Rb}}\frac{3\lambda^3}{4\pi^2} \cdot \frac{\Gamma}{\Omega_0} \cdot \frac{1}{\sqrt{\pi}u} \sum_{F_e=0}^{2} \sum_{m=-F_{\mathrm{g}}}^{F_{\mathrm{g}}} \frac{C_{1,m}^{F_{e,m\pm1}}}{a_{\pm}} \\
                   & \quad \times \int_{-\infty}^{\infty} dv e^{-(v/u)^2} \langle F_{e,m \pm 1}|\rho_{\mathrm{p}\mathrm{r}}|F_{\mathrm{g},m} \rangle,
    \end{aligned}\tag{4}
\end{gather}


\begin{equation}
    \mathbb{P}\left(\max_{\kappa N \leq k \leq (1 - \kappa)N} \pi \sqrt{\frac{\beta}{2}} \cdot  \frac{\rho_V (\gamma_k) N(\lambda_k - \gamma_k)}{\log N} \in [1 - \epsilon, 1 + \epsilon]\right) = 1 - o(1)
\end{equation}

\begin{equation}
    U_{n+1}(x) = \begin{bmatrix} V_n & \vec{\Phi}_n^t(x) \\ \vec{\Phi}_n (x) & 0 \end{bmatrix}, \quad \vec{\Phi}_n (x)= [\phi_1 (x), . . . ,\phi_n (x)]
\end{equation}


\begin{equation}
    B_{\sigma,\kappa}((w,W),(v,V)) = \int_{\Omega} \sigma \nabla w \cdot \nabla v dx + \int_{\partial\Omega} \kappa(W-w)(V-v) dS
\end{equation}

\begin{equation}
    a_j^- = \sqrt{\frac{m\Omega_j}{2\hbar}} \left( X_j + \frac{iP_j}{m\Omega_j} \right), \quad a_j^\dagger = \sqrt{\frac{m\Omega_j}{2\hbar}} \left( X_j - \frac{iP_j}{m\Omega_j} \right)
\end{equation}

\begin{equation*}
    =N \ln N - \sum_{i} a_i \ln \left( \frac{a_i}{\omega_i} \right)
\end{equation*}

\begin{equation}
    \mathbf{E}(\mathbf{r}) = \frac{\mathbf{F}(\mathbf{r})}{q_0} = \frac{1}{4\pi\epsilon_0} \frac{q}{r^2} \mathbf{\hat{u}}. \tag{I-2}
\end{equation}

\begin{equation}
    \mathbf{E}(\mathbf{r}) = \frac{1}{4\pi\epsilon_0} \sum_{l=1}^N \frac{q_l}{|\mathbf{r} - \mathbf{r}_l|^2} \mathbf{\hat{u}}_l. \tag{I-4}
\end{equation}

当然,以下是使用Markdown和LaTeX格式的答案:

这个问题要求证明向量函数的场线 \(y = y(x)\) 是一个微分方程的解。具体来说,需要展示 \(F(x, y) = iF_x(x, y) + jF_y(x, y)\) 是微分方程 \(\frac{dy}{dx} = \frac{F_y(x, y)}{F_x(x, y)}\) 的解。

这个问题的解决方法是使用链式法则。链式法则是微积分中的一个基本法则,它描述了复合函数的导数。在这个问题中,我们可以将 \(F(x, y)\) 看作是 \(x\) 和 \(y(x)\) 的函数,然后应用链式法则。

首先,我们计算 \(F(x, y)\) 的全导数:

$$
    dF = \frac{\partial F}{\partial x} dx + \frac{\partial F}{\partial y} dy
$$

这里,\(\frac{\partial F}{\partial x}\) 和 \(\frac{\partial F}{\partial y}\) 分别是 \(F\) 关于 \(x\) 和 \(y\) 的偏导数,它们分别等于 \(F_x(x, y)\) 和 \(F_y(x, y)\)。

然后,我们将 \(dy\) 表示为 \(dx\) 的函数:

$$
    dy = \frac{dy}{dx} dx
$$

将这个表达式代入 \(dF\) 的表达式,我们得到:

$$
    dF = F_x dx + F_y \frac{dy}{dx} dx
$$

由于 \(dF = 0\)(因为 \(F(x, y)\) 是常数),我们可以得到:

$$
    F_x + F_y \frac{dy}{dx} = 0
$$

从这个等式中解出 \(\frac{dy}{dx}\),我们得到:

$$
    \frac{dy}{dx} = -\frac{F_x}{F_y}
$$

这就证明了向量函数的场线 \(y = y(x)\) 是微分方程 \(\frac{dy}{dx} = \frac{F_y(x, y)}{F_x(x, y)}\) 的解。这是因为向量函数的场线的斜率在每一点上都等于该点的 \(F_y/F_x\)。这就是这个问题的解答。希望这个解答对你有所帮助!
\begin{equation}
    \iint_{S} \mathbf{E} \cdot \hat{\mathbf{n}} \, dS = \frac{q}{\varepsilon_0} \tag{II-1}
\end{equation}

\end{document}



